\section{ОХРАНА ТРУДА И БЕЗОПАСНОСТИ В ЧРЕЗВЫЧАЙНЫХ СИТУАЦИЯХ}

\subsection{Анализ условий труда в помещении офиса}

Помещение офиса находиться на первом этаже четырехэтажного железобетонного здания,
содержит одно рабочее место, которое включает: 
два ПК в комплектации: системный блок, LCD монитор, мышь, клавиатура.

Размеры помещения: длина -- 4 м, ширина -- 3 м, высота -- 2.5 м. Нормой, в соответствии с ДСанПiН 3.3.2-007-98, 
является площадь на одно рабочее место не менее 6,0 кв. м, объём -- не менее 20,0 куб. м, которая в данном случае
соблюдена. Помещение офиса содержит 1 рабочее место площадью 12 кв. м, объёмом 30 куб. м.

Приведем функциональную схему взаимодействия оборудования (рисунке \ref{fig:functional})

\begin{figure}[!ht]
    \centering
    \includegraphics[scale=0.6]{graphics/functional.png}
    \caption{Функциональная схема взаимодействия оборудования}
    \label{fig:functional}
\end{figure}

Проведем анализ системы <<Человек-Машина-Среда>> (Ч-М-С), структурная схема которой показана на рисунке 
\ref{fig:hme},
а направление связей приведено в таблице \ref{tab:hme}. Все элементы условно разделены на функциональные части:

\begin{description}
    \item [\(\text{Ч}_1\)] --- функциональная задача элемента -- управление <<машиной>> для получения предмета труда;
    \item [\(\text{Ч}_2\)] --- биологический обьект, непосредственно влияющий на среду за счет влаго-, тепло- и энерговыделения;
    \item [\(\text{Ч}_3\)] --- физиологическое состояние человека с учетом факторов, влияющих на него в процессе работы;
    \item [\(\text{М}_1\)] --- выполняет основную технологическую функцию (компиляция и сборка программы);
    \item [\(\text{М}_2\)] --- выполняет функцию аварийной защиты;
    \item [\(\text{М}_3\)] --- служит источником вредных влияний на человека и ОС;
    \item [\(\text{Среда}\)] --- внутренняя среда помещения: освещение, микроклимат и т.д.
    \item [\(\text{Предмет труда}\)] --- результат профессиональной деятельности человека (разработанный программный продукт).
\end{description}

\begin{figure}[!ht]
    \centering
    \includegraphics[scale=0.4]{graphics/hme.png}
    \caption{Система <<Человек-Машина-Среда>>}
    \label{fig:hme}
\end{figure}

{\small
\begin{center}
   \begin{longtable}{|p{1cm}|p{3cm}|p{12cm}|}
       \caption{Направление связей в системе <<Человек-Машина-Среда>>} \label{tab:hme} \\ \hline
       № & Связь & Содержание \\
       \endfirsthead

       \caption{Направление связей в системе <<Человек-Машина-Среда>> (продолжение)} \label{tab:hme} \\ \hline
       № & Связь & Содержание \\
       \endhead

       \hline
        1 & \(\text{Ч}_3-\text{С}\)   & Влияние оператора как биологического объекта на среду \\
       \hline
        2 & \(\text{С}-\text{Ч}_1\)   & Влияние окружающей среды на качество работы оператора \\
       \hline
        3 & \(\text{С}-\text{Ч}_3\)   & Влияние окружающей среды на состояние организма оператора \\
       \hline
        4 & \(\text{Ч}_1-\text{С}\)   & Информация про состояние среды, обрабатываемая оператором \\
       \hline
        5 & \(\text{М}_1, \ \text{М}_2, \ \text{М}_3-\text{Ч}_1\) & Информация про состояние ЭВМ обрабатываемая оператором \\
       \hline
        6 & \(\text{Ч}_1-\text{М}_1, \ \text{М}_2, \ \text{М}_3\) & Влияние оператора на управление ЭВМ и ее настройку \\
       \hline
        7 & \(\text{ПТ}-\text{М}_1\)  & Информация про объект труда, получаемая ЭВМ \\
       \hline
        8 & \(\text{М}_1-\text{ПТ}\)  & Влияние ЭВМ на объект труда \\
       \hline
        9 & \(\text{С}-\text{М}_1, \ \text{М}_2, \ \text{М}_3\)  & Влияние окружающей среды на работу ЭВМ \\
       \hline
       10 & \(\text{М}_1-\text{С}\)   & Влияние ЭВМ на среду \\
       \hline
       11 & \(\text{С}-\text{М}_3\)   & Информация про окружающую среду, получаемая ЭВМ \\
       \hline
       12 & \(\text{М}_3-\text{С} \)  & Целенаправленное влияние ЭВМ на среду \\
       \hline
       13 & \(\text{Ч}_3-\text{Ч}_1\) & Влияние состояние организма оператора на качество его работы \\
       \hline
       14 & \(\text{Ч}_3-\text{Ч}_2\) & Влияние физиологического состояния оператора на степень интенсивности обмена веществ
                                        между организмом и средой, и энерговыделение оператора \\
       \hline
       15 & \(\text{М}_2-\text{М}_1\) & Аварийное управляющее влияние \\
       \hline
       16 & \(\text{М}_1-\text{М}_2\) & Информация, необходимая для создания аварийных управляющих влияний \\
       \hline
       17 & \(\text{ПТ}-\text{Ч}_2 \) & Влияние предмета труда на психологическое состояние человека \\
       \hline
       18 & \(\text{Ч}_1-\text{Ч}_2\) & Влияние интенсивности работы программиста на физиологические функции \\
       \hline
   \end{longtable}
\end{center}
}

В помещении офиса, согласно ГОСТ 12.0.003-74, имеют место следующие опасные и вредные производственные факторы:
физические, психофизиологические. Химические и биологические факторы в данном помещении отсутствуют. 

К физическим факторам относятся: повышенный уровень шума на рабочем месте;
повышенное значение напряжения в электрической цепи, замыкание которой может
произойти через тело человека; повышенная или пониженная температура воздуха;
недостаток естественного света; недостаточная освещенность рабочей зоны. 

К психофизиологическим относятся: физические статические перегрузки; умственное перенапряжение; 
перенапряжение зрительных анализаторов; эмоциональные перегрузки.

Оценка факторов производственной среды и трудового процесса приведена в таблице \ref{tab:factors}.	

{\normalsize
\begin{longtable}[c]{|p{7cm}|p{2cm}|p{2cm}|p{0.5cm}|p{0.5cm}|p{0.5cm}|p{2cm}|}
    \caption{Оценка факторов производственной среды и трудового процесса} \label{tab:factors} \\

    \hline
\multirow{2}{*}{Фактор} & \multicolumn{2}{c|}{Значение фактора} & \multicolumn{3}{c|}{3 класс} & Действие, \\
\cline{2-6} & Норма & Факт & 1 ст. & 2 ст. & 3 ст. & \% смены \\ \hline
        \endfirsthead

        \caption{Оценка факторов производственной среды и трудового процесса (продолжение)} \label{tab:map} \\

        \hline
\multirow{2}{*}{Фактор} & \multicolumn{2}{c|}{Значение фактора} & \multicolumn{3}{c|}{3 класс} & Действие, \\
\cline{2-6} & Норма & Факт & 1 ст. & 2 ст. & 3 ст. & \% смены \\ \hline
    \endhead

    Шум, дБ(а)& \(\le 50\) & 50 & + & -- & -- & \\
    \hline
    Неионизирующее излучение: & & & & & & \\
   % а) промышленные частоты, \(\frac{\text{В}}{\text{м}}\)      & \(\le 25\)  & 1,2 & -- & -- & -- & 100 \\
    б) радиотехнический диапазон, \(\frac{\text{В}}{\text{м}}\) & \(\le 2,5\) & 1,0 & -- & -- & -- & 100 \\
    \hline
    Рентгеновское излучение, \(\frac{\text{мкР}}{\text{ч}}\)    & \(\le 100\) & 22          & -- & -- & -- & 80 \\
    \hline
    Микроклимат:                                                  &             &             &     &     &     &     \\
    а) температура воздуха, \(^{\circ}\mathrm{C}\)                & 22 -- 24    & 23          & --  & --  & --  & 100 \\
    б) скорость движения воздуха, \(\frac{\text{м}}{\text{с}}\)   & \(\le 0,1\) & \(\le 0,1\) & --  & --  & --  & 100 \\
    в) относительная влажность, \%                                & 40 -- 60    & 60          & --  & --  & --  & 100 \\
    \hline
    Освещение:            &               &     &     &     &     &    \\
    а) естественное, КЕО  & \(\ge 1,5\%\) & 4\% & --  & --  & --  & 70 \\
    б) искусственное, лк  & 300--500      & 250 &  +  & --  & --  & 30 \\
    \hline
    Тяжесть труда:         &               &     &     &     &     &    \\
    а) мелкие стереотипные
       движения, кол-во за
       смену                          & \(\le 40000\) & 45000 & +   & -- & -- & 100 \\
    б) рабочая поза, \%               & \(\le 25\)    & 20    & --  & -- & -- & 20  \\
    в) перемещение в пространстве, км & \(\le 10\)    & 1     & --  & -- & -- & 5   \\
    \hline
    Напряжённость труда: время наблюдения
    за ходом производственного процесса, ч & \(2 \div 3\) & 4 & + & -- & -- & 90 \\
    \hline
    Сменность: & 2-х сменная работа без ночной смены & 2-х сменная работа без ночной смены & -- & -- & -- & -- \\
    \hline
    Количество факторов: &  &  & 3 & -- & -- & \\
    \hline
\end{longtable}
}
По результатам гигиенической оценки условий труда было выявлено, что условия и характер труда относятся
к III классу 1 степени вредности, и не соответствуют требованиям гигиенической классификации.
Так как доминирующим вредным фактором является недостаточная освещенность рабочей зоны, то для снижения
его вредного влияния произведен выбор и расчет системы искусственного освещения помещения офиса. 

\subsection{Промышленная безопасность в помещении офиса}

По степени опасности поражения электрическим током, помещение офиса относится к классу помещений без повышенной
опасности согласно НПАОП 40.1-1.21-98.

Электроснабжение оборудования осуществляется от трехфазной четырехпроводной сети с глухозаземленной нейтралью,
ток переменный, напряжение 380/220 В. В соответствии с требованиями НПАОП 40.1-1.32-01 для электрооборудования
переменного тока напряжением до 1000 В используется зануление для устранения опасности поражения людей электрическим
током при пробое на корпус. 

Сопротивление изоляции сети не менее 0,5 МОм, сопротивление одиночного повторного
заземления нулевого провода не более 30 Ом согласно требованиям НПАОП 40.1-1.32-01. 

Электросеть розеток для питания ПЭВМ проложена под съемным полом в гибких металлических рукавах.

Согласно НПАОП 0.00-4.12-05 проводится вводный, первичный на рабочем месте, повторный, а при необходимости -- 
внеплановый и целевой инструктажи.

\subsection{Производственная санитария в офисном помещении}

Так как работа производится сидя и не требует физического напряжения (энергозатраты организма до120 ккал\slashч),
она относится к работам категорий I а согласно ДСН 3.3.6-042-99. Оптимальные параметры микроклимата,
согласно ДСН 3.3.6.042-99, приведены в таблице \ref{tab:clima}.
\begin{table}[!ht]
    \centering
    \caption{Оптимальные параметры микроклимата}
    \label{tab:clima}

    \begin{tabular}{|p{2.5cm}|p{4.5cm}|p{4cm}|p{4cm}|}
\hline
Период года & Температура воздуха, \(^{\circ}\mathrm{C}\) & Относительная влажность, \% & Скорость движения воздуха \\
\hline
Холодный    & 22 -- 24                                    & 40 -- 60                    & \(\le 0,1\) \\
\hline
Тёплый      & 23 -- 25                                    & 40 -- 60                    & \(\le 0,1 \) \\
\hline
    \end{tabular}
\end{table}
Для поддержания указанных условий труда в теплый период года в помещении применяется кондиционер,
а в холодный период года осуществляется отопление. 

Для повышения работоспособности и предупреждения утомления работников умственного труда  устанавливается
рациональный режим труда и отдыха, правильно организуется рабочее место согласно эргономическим  требованиям
ГОСТ 12.2.032-78. Схема размещения рабочего места показана на рисунке \ref{fig:workplace}.

\begin{figure}[!ht]
    \centering
    \includegraphics[scale=0.5]{graphics/workplace2.png}
    \caption{Схема размещения рабочего места и план эвакуации}
    \label{fig:workplace}
\end{figure}

Обозначения для рисунка \ref{fig:workplace} :

\begin{enumerate}
    \item -- Огнетушитель ВВК-1,4;
    \item -- Стул;
    \item -- ПК;
    \item -- Стол.
\end{enumerate}

Для нормализации искусственного освещения рабочих зон необходима его реконструкция. Таким образом,
необходимо произвести выбор типа светильников и расчет их размещения в данном помещении.

Для искусственного освещения офиса рекомендуется использовать люминесцентные лампы, у которых большая световая отдача,
малая яркость светящейся поверхности, близкий к естественному спектральный состав излучения. Требуемая освещенность --
300 лк согласно ДБН В.2.5-28-2006.

Для данного типа помещений применяют потолочные светильника типа ЛВО 4Х18, имеющие следующие габариты 595Х595Х72мм.
В указанных светильниках применяются ЛЛ, в которых Фл = 1150лм.

Расчет освещенности производится методом коэффициента использования светового потока [1].
Расчет сводится к определению числа рядов светильников выбранного типа и числа светильников в ряду.

Расчетным методом является

\begin{equation}\label{eq:main}
    N = \dfrac{E_{\text{н}} \times K_{\text{з}} \times S \times z}{n \times \text{Ф}_{\text{св}} \times \eta \times g},
\end{equation} 
\begin{enumerate}[  ]
    \item где $E_{\text{н}}$ - нормируемая минимальная освещенность, $E_{\text{н}}$ = 300 лк;
    \item $K_{\text{з}}$ - коэффициент запаса, учитывающий запыление светильников и износ источников света в процессе эксплуатации;
    \item S - площадь пола помещения, ${\text{м}}^2$; 
    \item z - коэффициент неравномерности освещения;
    \item n - число рядов светильников, определяемое из условия наивыгоднейшего соотношения $x = L \slash h$ (x = 1.4)
          (L - расстояние между рядами светильников(м); h -- высота подвеса светильников над рабочей поверхностю(м));
    \item g - коэффициент затенения (при отсутствии затенения g = 1); 
    \item $\eta$ - коэффициент использования светового потока; 
    \item ${\text{Ф}}_{\text{св}}$ - световой поток, излучаемый светильником, лм; 
\end{enumerate}

При условии чистки светильников не реже двух раз в месяц $K_{\text{з}} = 1.4$. 
Для люминесцентных ламп z = 1.1. Коэффициент использования светового потока $\eta$ зависит от типа светильников,
коэффициентов отражения светового потока от стен $r{\text{c}}$, потолка $r{\text{п}}$ и пола $r{\text{пола}}$,
а также индекса помещения i

\begin{equation}\label{eq:index}
    i = \dfrac{A \times B}{h \times (A + B)},
\end{equation}
\begin{enumerate}[  ]
    \item где A -- длина помещения;
    \item B -- ширина помещения;
    \item h -- высота подвеса светильников над рабочей поверхностью, м.
\end{enumerate}

Высота подвеса светильников над рабочей поверхностью

\begin{equation}\label{eq:high}
    h = H - H_{\text{рз}} - H_{\text{св}},
\end{equation}
\begin{enumerate}[  ]
    \item где H -- высота помещения, м;
    \item $H_{\text{рз}}$ -- высота рабочей поверхности над полом, $H_{\text{рз}}$ = 0.8 м;
    \item $H_{\text{св}}$ -- высота свеса светильника, $H_{\text{св}}$ = 0.4 м.
\end{enumerate}

Расстояние между стенами и крайними рядами светильников рассчитывается следующим образом

\begin{equation}\label{eq:length}
    l = (0.3 \ldots 0.5) \times L = 0.3 \times L.
\end{equation}
Принимаем $r{\text{c}}$ = 50\%, $r{\text{п}}$ = 70\%,  $r{\text{пола}}$ = 10\% для данного помещения.
Подставим данные в формулы (1.1-1.4) и получим следующие результаты
\[
    h = 2.5 - 0.8 - 0.4 = 1.3 (\text{м});
\]
\[
    L = 1.4 \times 1.3 = 1.82 (\text{м});
\]
\[
    l = 0.3 \times 1.82 \approx 0.5 (\text{м});
\]
\[
    n = \dfrac {1.82}{1.4} \approx 2;
\]
\[
    i = \dfrac {12}{1.3 \times 7} \approx 1.3;
\]
\[
    \eta = 0.48 [1].
\]

Поток, излучаемый светильником: ${\text{Ф}}_{\text{св}} = 4 \times 1150 = 4600$ лм.

По приведенной выше формуле \eqref{eq:main} определяем количество светильников в ряду
\[
    N = 300 \times 1.4 \times 12 \times {\dfrac{1.2}{2}} \times 4600 \times 0.48 \times 1 = 1 (\text{шт}).
\]
Таким образом, согласно расчетам для обеспечения оптимального уровня освещенности необходимо 2 ряда светильников по
одному в каждом ряде. Схема расположения светильников изображена на рисунке \ref{fig:torch}.

\begin{figure}[!ht]
    \centering
    \includegraphics[scale=0.4]{graphics/torch.png}
    \caption{Схема расположения светильников}
    \label{fig:torch}
\end{figure}

\subsection{Безопасность в чрезвычайных ситуациях}
Для данного офисного помещения наболее вероятной чрезвычайной ситуацией является пожар.
Поэтому далее будут рассмотрены вопросы пожарной безопасности здания.

Согласно НАПБ Б.03.002-2007 помещение офиса относится к категории B,
так как в нем находятся горючие вещества и материалы. 

Помещение офиса относится к классу П-IIа в соответствии с НПАОП 40.1-1.21-98, 
так как в помещении находятся твердые горючие вещества (системный блок, монитор). В офисе присутствуют 
вещества и материалы, которые могут гореть (бумага, пластмасса, паркет). Здание по степени огнестойкости,
согласно ДБНВ.1.1.7-2002, относится к I степени огнестойкости. 

Причинами воспламенения в помещении могут служить: искрение в коммутационном оборудовании, 
замыкания в электрической сети, нарушение правил пожарной безопасности. 

Для предупреждения пожара необходимо проводить ряд технических и организационных мероприятий в соответствии 
с ГОСТ 12.1.004 – 91.

В системе предотвращения пожара предусмотрено:
    коммутация проводов выполнена разъемами;
    сечение проводов выбрано в соответствии с максимальным током нагрузки;
    использование максимально негорючих веществ и материалов.

Система противопожарной защиты: помещение оборудовано двумя огнетушителем ВВК-1,4 согласно НАПБ Б.03.001-2004  из расчета один огнетушитель на 3 ПЭВМ,
но не меньше двух в помещении, а также электрической пожарной сигнализацией согласно НАПБ Б.03.001-2004 из расчета 
не менее двух в помещении, состоящей из комбинированных тепловых и дымовых извещателей типа КИ–1 в количестве 2 шт.,
температура срабатывания которых составляет 50–80$^{\circ}\mathrm{C}$.

Для успешной эвакуации персонала при пожаре размеры дверей рабочего помещения имеют следующие габариты: 
ширина  – 1,5 м, высота -  2,0 м, а ширина коридора - 1,8 м.  При возникновении пожара осуществляется эвакуация людей, 
находящихся в помещении, в соответствии с планом эвакуации (рисунок \ref{fig:workplace}).

